.
\pagebreak

\noindent\parbox[l][71mm]{80mm}
{
	\begin{center}
		{ \textbf{Садоводческое некоммерческое\\ товарищество
				№  2 Акционерного\\ общества "Югтекс"\\
		}}
		\footnotesize{г. Краснодар, Ростовское шоссе, 13 км\\
			yugteks@gmail.com, info@yugteks2.ru\\
			тел. 8(918) 451-66-11\\
			Почтовый адрес: 350032, г. Краснодар,\\ пос. Лазурный, ул. Октябрьская, 2 А, а/я   900
		}\\
		{ИНН 2311046795 КПП 231201001 ОГРН 1032307173018}
	\end{center}
	\hbox to 6cm{ }}\hfill
\parbox[l][71mm]{65mm}
{ \begin{center}
		{
			\textbf{\комуФИО}\\
			\vspace{5mm}
			{\footnotesize{350032, г. Краснодар,\\ п. Лазурный, СНТ № 2 АО "Югтекс", \\
			ул. \УЛИЦА, \УЧАСТОК\\
		e-mail: \email } }
			
		}
	\end{center}
	\hbox to 6cm{ }}
%%%%%%%%%%%%%%%%%%%%%%%%%%%%%%%%%%%%%%%%%%%%%%%%%%%%%%%%%%%%%%%%
\linebreak
\vspace{-12mm}

\underline{исх. № \исхВладелец \, от \py{dat}} 

\vspace{2mm}
\begin{center}
	\textbf{УВЕДОМЛЕНИЕ 
		%	члена СНТ № 2 АО «Югтекс»
		ОБ ОГРАНИЧЕНИИ ПОДАЧИ ЭЛЕКТРОЭНЕРГИИ}
\end{center}
\par
%\vspace{2mm}

\begin{center}
	Уважаемый(ая) \textbf{ \ФИО}!
\end{center}
\vspace{-3mm} Вы являетесь членом СНТ № 2 АО «Югтекс» (далее – Товарищество), в связи с чем являетесь плательщиком обязательных членских и целевых взносов, состоящих из денежных средств, периодически вносимых членами Товарищества на оплату труда работников, заключивших трудовые договоры с Товариществом, части стоимости электрической энергии, потребленной при использовании имущества общего пользования Товарищества, %части потерь электрической энергии, возникших в объектах электросетевого хозяйства, принадлежащих Товариществу,
другие текущие расходы Товарищества. 
Согласно Уставу Товарищества членские взносы вносятся членами Товарищества
%/правообладателем земельного участка, находящегося в пределах территории Товарищества, безналичным способом на расчетный счет Товарищества 
ежегодно, до даты, установленной решением общего собрания членов садоводческого товарищества.
%(до 31  декабря 2022 года).
По состоянию на \py{dat} за Вами образовалась задолженность по внесению членских взносов в размере \задолженность (\числопрописью{\задолженность})  руб., совокупная задолженность по оплате членских и целевых взносов составляет всего \задолженностьсовокупная (\числопрописью{\задолженностьсовокупная}) руб. Последний платеж был произведен \последнийплатеж г.
Согласно абз. 5 пп. б п. 2 Правил полного и (или) частичного ограничения режима потребления электрической энергии, утвержденных постановлением Правительства РФ от 04.05.2012 № 442, в случае возникновение у граждан, ведущих садоводство или огородничество на земельных участках, расположенных в границах территории садоводства или огородничества, задолженности по оплате электрической энергии по договору энергоснабжения или перед садоводческим или огородническим некоммерческим товариществом ввиду неисполнения или ненадлежащего исполнения обязательств по оплате части стоимости электрической энергии, потребленной при использовании имущества общего пользования садоводческого или огороднического некоммерческого товарищества, и части потерь электрической энергии, возникших в объектах электросетевого хозяйства, принадлежащих садоводческому или огородническому некоммерческому товариществу, вводится ограничение режима потребления.
В соответствии с п. 17.1 Правил полного и (или) частичного ограничения режима потребления электрической энергии, утвержденных постановлением Правительства РФ от 04.05.2012 № 442, введение ограничения режима потребления в отношении граждан - потребителей коммунальной услуги по электроснабжению осуществляется по основаниям и в порядке, которые установлены жилищным законодательством Российской Федерации, а именно Правилами предоставления коммунальных услуг собственникам и пользователям помещений в многоквартирных домах и жилых домов, утвержденными постановлением Правительства Российской Федерации от 06.05.2011 № 354.
Данный порядок, как и общий порядок, предусмотренный в Правилах полного и (или) частичного ограничения режима потребления электрической энергии, утвержденных постановлением Правительства РФ от 04.05.2012 № 442, подразумевает предварительное уведомление гражданина о предстоящем ограничении, предоставление срока для погашения образовавшейся задолженности (20 дней), частичное, а затем полное ограничение режима потребления электроэнергии. 
Согласно подп. а п. 117 Правил предоставления коммунальных услуг собственникам и пользователям помещений в многоквартирных домах и жилых домов, утвержденными постановлением Правительства Российской Федерации от 06.05.2011 № 354 в случае неполной оплаты потребителем коммунальной услуги в порядке и сроки, которые установлены названными Правилами, исполнитель ограничивает или приостанавливает предоставление коммунальной услуги, предварительно уведомив об этом потребителя.
При этом под неполной оплатой потребителем коммунальной услуги понимается наличие у потребителя задолженности по оплате 1 коммунальной услуги в размере, превышающем сумму 2 месячных размеров платы за коммунальную услугу, исчисленных исходя из норматива потребления коммунальной услуги независимо от наличия или отсутствия индивидуального или общего прибора учета и тарифа (цены) на соответствующий вид коммунального ресурса, действующих на день ограничения предоставления коммунальной услуги (п. 118 Правил предоставления коммунальных услуг собственникам и пользователям помещений в многоквартирных домах и жилых домов, утвержденными постановлением Правительства Российской Федерации от 06.05.2011 № 354).
Согласно подп. в.1 п. 4 Правил полного и (или) частичного ограничения режима потребления электрической энергии, утвержденных постановлением Правительства РФ от 04.05.2012 № 442 Ограничение режима потребления вводится по инициативе садоводческого или огороднического некоммерческого товарищества – в связи с наступлением обстоятельств, указанных в абзаце шестом подпункта «б» пункта 2 настоящих Правил.
Ежегодное внесение членских взносов в Товарищество включает в себя 12 месячных размеров платы за электрическую энергию, потребленную при использовании имущества общего пользования Товарищества. 
На момент       \py{dat}  у вас имеется задолженность в размере   \задолженность (\числопрописью{\задолженность}) руб. за неисполнение и/или ненадлежащее исполнение обязанности по внесению  обязательных членских взносов. 
Если Вами до \py{middata}      %    [число, месяц, год]
 %(за два дня до даты отключения)
  не будет произведено погашение задолженности в размере \задолженность (\числопрописью{\задолженность})            рублей, то Вы обязаны до 12 часов 00 минут \py{middata} г.        %[число, месяц, год] 
  самостоятельно полностью ограничить режим потребления электрической энергии. 
Настоящим уведомляем Вас о том, что \py{middata} \, в 14 часов 00 минут     территориальной сетевой организацией будет осуществлено частичное ограничение режима потребления электрической энергии, а \py{enddata} \, в 10 часов 00 минут полное ограничение режима потребления электрической энергии.
Соответствующая заявка в территориальную сетевую организацию направлена  \py{dat}.

\vspace{15mm}

%\noindent Приложения:\\
%1. Копия технических условий 
%\vspace{7mm}
\noindent Председатель СНТ № 2 АО "Югтекс" \hfill    \rule{3cm}{0.1 mm}    Мраморнов А.В.

%\подписьспечатью

\pagebreak
