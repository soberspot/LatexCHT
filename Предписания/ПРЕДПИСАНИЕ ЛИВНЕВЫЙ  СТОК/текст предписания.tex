\begin{center}
	Уважаемый(ая) \textbf{ \ФИО}!
\end{center}
\vspace{-3mm}
%Вы являетесь владельцем садового участка, расположенного в садоводческом некоммерческом товариществе № 2 АО «Югтекс» (далее – Товарищество) по адресу: \адресвснт.  
%Комиссионной проверкой, проведенной \датапроверки 
Правлением товарищества выявлено наличие  отвода дождевой воды  от строений, расположенных на Вашем участке, на дороги и земли общего пользования  Товарищества.  


Руководствуясь Уставом и  Правилами внутреннего распорядка садоводческого некоммерческого товарищества  № 2 АО <<Югтекс>>, утвержденными решением общего собрания Протокол № 3 от 31.10.2018 г. и Протоколом № 1 от 15.11.2020 г., п. 5.2.5.,  5.3.8,  ГК РФ Статья 210,  предлагаю  Вам, как владельцу недвижимого имущества, расположенного по адресу: г. Краснодар, садоводческое некоммерческое товарищество № 2 АО <<Югтекс>>,    улица \textbf{\УЛИЦА} \. участок №~\textbf{\УЧАСТОК} \ устранить самовольно устроенный отвод осадочных вод на  земли общего пользования (дороги) в срок  до %\py{data_ispolnenya}~г.
«\rule{10mm}{0.1 mm}»\rule{35mm}{0.1 mm} 2023 года.  

Отвод  дождевых вод рекомендуется производить в ливневые дренажные колодцы, канавы, самостоятельно устроенные в границах Вашего участка. 

%Результатом отведения дождевой воды с Вашего участка является повреждение  п%Оценочная стоимость устранения повреждения в текущих рыночных ценах составляет \rule{8mm}{0.1 mm}  рублей.окрытия дорог общего пользования.  

%
%в виде ямы  площадью  \rule{8mm}{0.1 mm}~$m^2$ объемом \rule{8mm}{0.1 mm}~$m3$ (фотоизображение повреждений прилагается).  

%Правление товарищества предлагает Вам в десятидневный срок перекрыть отвод дождевых вод на земли общего пользования,  самостоятельно устранить повреждение дорожного покрытия в тридцатидневный срок или произвести компенсацию затрат на устранение повреждений силами и средствами товарищества.  В противном случае согласно решения Общего собрания от 01.01.1111 в целях компенсации ущерба Товарищество  будет вынуждено прибегнуть к взысканию с владельца участка  денежных  средств, необходимых для устранения повреждений дорожного покрытия из расчета рыночной стоимости затраченного инертного материала, стоимости минимального времени аренды дорожной спецтехники, стоимости оценки ущерба и судебных расходов. 


%В случае невыполнения данного предписания, руководствуясь Правилами внутреннего распорядка п.3.5. «В случае причинения ущерба собственности СНТ (дорога, заборы и т.п.) нарушитель производит их ремонт за свой счет по заявлению пострадавшей стороны либо возмещает ущерб»; п.5.1.2 «Установить возмещение материального ущерба с виновного лица в размере, определяемом расчетным путем по рыночным ценам услуг, необходимых для ликвидации последствий», Товарищество будет вынуждено обратиться в суд для решения вопроса о проведении демонтажа слива в принудительном порядке, возложении затрат на владельца участка и взыскании причиненного ущерба.

В случае невыполнения данного предписания в указанные сроки, лицо, допустившее нарушение  производит ремонт повреждений участка дороги, образовавшихся вследствие воздействия выявленного стока воды, за свой счет или возмещает ущерб в размере, определяемом расчетным путем по рыночным ценам услуг и материалов, необходимых для устранения повреждений.

Товарищество оставляет за собой право обратиться в суд для решения вопроса о проведении демонтажа слива в принудительном порядке с возложением затрат и судебных издержек на владельца участка   и взыскании причиненного ущерба.

%вреда имуществу истца и его оценки в материальном выражении

%В дальнейшем рекомендуем  отвод  дождевых вод производить в ливневые дренажные колодцы, канавы, устроенные в границах Вашего участка. 

 
% код на python:
%\begin{pycode}
%    username = str(fio)
%    usernamelist = username.split(' ')
%    print(f'{usernamelist[0]}  {usernamelist[1][0:1]}. {usernamelist[2][0:1]}.')
%\end{pycode}

%\begin{pycode}
%from datetime import timedelta, datetime
%
%
%d1 = datetime.strptime("01.01.2023", "%d.%m.%Y")
%
%in_y_days = d1 + timedelta(7)
%
%print(in_y_days.strftime("%d.%m.%Y"))
%\end{pycode}

%\pyc{print(fio)}

