\begin{center}
	Уважаемый  \ФИО!
\end{center}
%
%Вы являетесь владельцем садового участка, расположенного в садоводческом некоммерческом товариществе № 2 АО «Югтекс» (далее – Товарищество) по адресу: \адресвснт.  
%Комиссионной проверкой, проведенной \датапроверки 
Правлением товарищества установлено наличие  отвода дождевой воды  от строений, расположенных на Вашем участке, на земли общего пользования Товарищества.  


Руководствуясь Правилами внутреннего распорядка СНТ № 2 АО «Югтекс», утвержденными решением общего собрания Протокол № 3 от 31.10.2018 г. и Протоколом № 1 от 15.11.2020 г., п. 5.2.5. «Пользоваться землей только в границах отведенного ему Садового участка, не допуская его увеличения и использования земли за его границами»; 5.3.8. «Не нарушать покрытия дорог...», Правление СНТ № 2 АО «Югтекс» уведомляет Вас, как владельца земельного участка, расположенного по адресу:                                            улица \rule{5cm}{0.1 mm} № \rule{1cm}{0.1 mm} , о  необходимости осуществления демонтажа самовольно возведенного сооружения (слива) на принадлежащие товариществу земли общего пользования (дороги общего пользования) в срок  до «\rule{8mm}{0.1 mm}»\rule{4cm}{0.1 mm} 2023 года.  


%
%Длительное воздействие воды и влаги отрицательно влияет на технические и эксплуатационные параметры гравийного покрытия дорог.  
Согласно решений Общих собраний Товарищества от 31.10.2018 и 15.11.2020 г.г. владелец садового участка обязан не нарушать покрытия дорог, не вывозить с полотна дорог гравий, отсев, песок и т.д. 

%Результатом отведения дождевой воды с Вашего участка является повреждение  п%Оценочная стоимость устранения повреждения в текущих рыночных ценах составляет \rule{8mm}{0.1 mm}  рублей.окрытия дорог общего пользования.  

%
%в виде ямы  площадью  \rule{8mm}{0.1 mm}~$m^2$ объемом \rule{8mm}{0.1 mm}~$m3$ (фотоизображение повреждений прилагается).  

%Правление товарищества предлагает Вам в десятидневный срок перекрыть отвод дождевых вод на земли общего пользования,  самостоятельно устранить повреждение дорожного покрытия в тридцатидневный срок или произвести компенсацию затрат на устранение повреждений силами и средствами товарищества.  В противном случае согласно решения Общего собрания от 01.01.1111 в целях компенсации ущерба Товарищество  будет вынуждено прибегнуть к взысканию с владельца участка  денежных  средств, необходимых для устранения повреждений дорожного покрытия из расчета рыночной стоимости затраченного инертного материала, стоимости минимального времени аренды дорожной спецтехники, стоимости оценки ущерба и судебных расходов. 


В случае невыполнения данного предписания, руководствуясь Правилами внутреннего распорядка п.3.5. «В случае причинения ущерба собственности СНТ (дорога, заборы и т.п.) нарушитель производит их ремонт за свой счет по заявлению пострадавшей стороны либо возмещает ущерб»; п.5.1.2 «Установить возмещение материального ущерба с виновного лица в размере, определяемом расчетным путем по рыночным ценам услуг, необходимых для ликвидации последствий», Товарищество будет вынуждено обратиться в суд для решения вопроса о проведении демонтажа слива в принудительном порядке, возложении затрат на владельца участка и взыскании причиненного ущерба.

%В дальнейшем рекомендуем  отвод  дождевых вод производить в ливневые дренажные колодцы, канавы, устроенные в границах Вашего участка. 

\vspace{3mm}

%\noindent Приложения:\\
%1. Копия технических условий 


\vspace{3mm}
\noindent Комиссия в составе:\\

