\documentclass[12pt]{article}
% пример использования \FloatBarrier из placeins
% demonstration of \FloatBarrier

% СРАВНИТЕ ПОВЕДЕНИЕ С И БЕЗ placeins ЗАКОМЕНТИРОВАВ СТРОКУ
%   С  \usepackage[...]{placeins}
% 
% для автоматического \FloatBarrier до каждой \section:
% if you want each section to be a barrier, use 
\usepackage[section,above,below]{placeins}


% команда \afterpage
% provides \afterpage command
\usepackage{afterpage}


% На случай, если без \FloatBarrier - заглушка
\providecommand\FloatBarrier{}

%\usepackage{graphicx} % we do not need it for demonstration

\usepackage{lipsum}
\newcommand\fakepic[2]{\framebox[#1cm]{\rule{0pt}{#2cm}\small #1$\times$#2 cm$^2$}}


\begin{document}

\section{First section}\label{sec:1}

Demonstration of FloatBarrier from placeins. When italic text starts,
all the float should be already in place. One could use a clearpage
right before the italic text, but that would not only output
all the floats but also start a new page. FloatBarrier just empties
the floats que and then proceeds with typesetting the text without
creating a new page. 


\lipsum[1]

\begin{figure}
\centering
\fakepic{5}{1}
\caption{Figure from section \ref{sec:1}}\label{fig:1}
\end{figure}


\lipsum[1]

\begin{figure}
\centering
\fakepic{5}{6}
\caption{Figure from section \ref{sec:1}}\label{fig:2}
\end{figure}

\begin{figure}
\centering
\fakepic{5}{6}
\caption{Figure from section \ref{sec:1}}\label{fig:3}
\end{figure}

\begin{figure}[b]
\centering
\fakepic{5}{6}
\caption{Figure from section \ref{sec:1}}\label{fig:4}
\end{figure}

\section{qqqq www eeeee rrrr}\label{sec:2}

\lipsum[1]

\begin{figure}
\centering
\fakepic{5}{6}
\caption{Figure from section \ref{sec:2}}\label{fig:5}
\end{figure}

\FloatBarrier
{\itshape
When this italic text starts, all the floating figures 
up to Figure \ref{fig:5} are already 
typeset. The text will go (if possible) on the same page as the last 
figure.}

\begin{figure}
\centering
\fakepic{5}{6}
\caption{This figure does not fit on the page where it could be started. 
If you used FloatBarrier right after it, it would lead to a partly
empty page. But using afterpage prevents that.}\label{fig:6}
\end{figure}

% \FloatBarrier is not needed here. But if you decided to use
% due to some reason, do it within \afterpage
\afterpage{\FloatBarrier}

\lipsum[1-2]



\end{document}
