\noindent\parbox[l][71mm]{80mm}
{
	\begin{center}
		{\small \textbf{Садоводческое некоммерческое\\ товарищество
				№  2 Акционерного\\ общества "Югтекс"\\
		}}
		\footnotesize{г. Краснодар, Ростовское шоссе, 13 км\\
			yugteks@gmail.com, info@yugteks2.ru\\
			тел. 8(918) 451-66-11\\
			Почтовый адрес: 350032, г. Краснодар,\\ пос. Лазурный, ул. Октябрьская, 2 А, а/я   900
		}\\
		{ИНН 2311046795 КПП 231201001 ОГРН 1032307173018}
	\end{center}
	\hbox to 6cm{ }}\hfill
\parbox[l][71mm]{65mm}
{ \begin{center}
		\small{
			\textbf{Диретору\\ ООО "Краснодарэнерго"}\\
			\vspace{3mm}
			Л. В. Самарченко\\
			\vspace{3mm}
			{\footnotesize Адрес: 350028, г. Краснодар,\\ ул. им. Героя Сарабеева В.И., д. 5, офис 10}
			
		}
	\end{center}
	\hbox to 6cm{ }}
%%%%%%%%%%%%%%%%%%%%%%%%%%%%%%%%%%%%%%%%%%%%%%%%%%%%%%%%%%%%%%%%
\linebreak
\vspace{-12mm}

\underline{исх. № \исх  от \py{dat} } 

\vspace{8mm}
\begin{center}
	\Large\textbf{Заявка на ограничение режима потребления электроэнергии}
\end{center}
\par
\vspace{10mm}



\ФИО является членом СНТ № 2 АО «Югтекс» (далее – Товарищество)/правообладателем земельного участка, находящегося в пределах территории Товарищества, в связи с чем на основании ст. 5 и ст. 11    Федерального закона  "О ведении гражданами садоводства и огородничества для собственных нужд и о внесении изменений в отдельные законодательные акты Российской Федерации"   от 29.07.2017 N 217-ФЗ является плательщиком обязательных членских взносов, состоящих из денежных средств, периодически вносимых членами Товарищества на оплату труда работников, заключивших трудовые договоры с Товариществом, части стоимости электрической энергии, потребленной при использовании имущества общего пользования Товарищества,
% части потерь электрической энергии, возникших в объектах электросетевого хозяйства, принадлежащих Товариществу, 
другие текущие расходы Товарищества. 
Согласно Уставу Товарищества членские взносы вносятся членами Товарищества/правообладателем земельного участка, находящегося в пределах территории Товарищества, безналичным способом на расчетный счет Товарищества ежегодно, до даты, установленной решением общего собрания членов садоводческого товарищества    (до 31  декабря 2022 года). 
По состоянию на  \py{dat}   за \заФИО образовалась задолженность по внесению членских взносов в размере \задолженность (\числопрописью{\задолженность}) рублей.
Согласно абз. 5 пп. б п. 2 Правил полного и (или) частичного ограничения режима потребления электрической энергии, утвержденных постановлением Правительства РФ от 04.05.2012 № 442, в случае возникновение у граждан, ведущих садоводство или огородничество на земельных участках, расположенных в границах территории садоводства или огородничества, задолженности по оплате электрической энергии по договору энергоснабжения или перед садоводческим или огородническим некоммерческим товариществом ввиду неисполнения или ненадлежащего исполнения обязательств по оплате части стоимости электрической энергии, потребленной при использовании имущества общего пользования садоводческого или огороднического некоммерческого товарищества, и части потерь электрической энергии, возникших в объектах электросетевого хозяйства, принадлежащих садоводческому или огородническому некоммерческому товариществу, вводится ограничение режима потребления.
Согласно подп. в.1 п. 4 Правил полного и (или) частичного ограничения режима потребления электрической энергии, утвержденных постановлением Правительства РФ от 04.05.2012 № 442, ограничение режима потребления вводится по инициативе садоводческого или огороднического некоммерческого товарищества – в связи с наступлением обстоятельств, указанных в абзаце шестом подпункта «б» пункта 2 настоящих Правил.
С учетом вышеизложенного,  прошу в 14 часов 00 минут  \py{middata} осуществить ограничение режима потребления электрической энергии в отношении прибора учета  № \счетчик , принадлежащего \комуФИО, лицевой счет № \лс. 

\vspace{5mm}

%\noindent Приложения:\\
%1. Копия технических условий 


\vspace{15mm}
\noindent Председатель СНТ № 2 АО "Югтекс" \hfill    \rule{3cm}{0.1 mm}    Мраморнов А.В.